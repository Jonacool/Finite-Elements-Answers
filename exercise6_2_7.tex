\documentclass{article}
\usepackage{titlesec}
\usepackage{titling}
\setcounter{secnumdepth}{0} %Cancels the enumerating of sections
\usepackage[nottoc]{tocbibind} %Adds bibliography to table of contents
\usepackage[margin=1.1in]{geometry}
\usepackage[english]{babel}
\usepackage[]{amsmath, amssymb}
\usepackage[]{parskip}
\usepackage[]{graphicx}
\usepackage[]{enumerate}
\usepackage[]{url}
\usepackage[]{float}
\usepackage[hidelinks]{hyperref}
\usepackage{graphicx}
\usepackage[]{color}
\usepackage{listings}
\usepackage{apacite}
\usepackage{caption}
\usepackage{gensymb}
\usepackage{makecell}
\usepackage{booktabs}
\usepackage{physics}

\title{Exercise 6.2.7}
\author{Jonathan Pilgram}
\date{\today}

\begin{document}
\maketitle

Integration rules can be written in the general form of: 
\[
	\int_{x_{k-1}}^{x_k} g(x) \mathrm{d}x = \sum_{k=1}^{r} w_k g(v_k)
\] 
With $r, w_k, v_k$ the number of quadrature points, the weights and the quadrature points respectively. 
\\\\
For the midpoint rule $r=1, w_k = 1, v_k = x_{k-3/2}  $\\
For the trapezoid rule: $r=2, w_k = \frac{1}{2}, v_k = x_{ k-2 } $ \\
For Simpson's rule: $r=3, w_k = \frac{4}{6} (\frac{1}{1+(k-2)^2} ) ^2, v_k=x_{ \frac{k}{2} - \frac{3}{2}  } $

\end{document}
